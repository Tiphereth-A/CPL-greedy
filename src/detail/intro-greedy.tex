\begin{frame}[fragile]{简介}

	\begin{itemize}
		\item<1-> 贪心算法 (greedy algorithm), 又称贪婪算法, 是一种在每一步选择中都采取在当前状态下最好或最优 (即最有利) 的选择, 从而希望导致结果是最好或最优的算法
		\item<2-> 贪心算法在有 \textbf{最优子结构}\xspace 的问题中尤为有效. 简单地说, 对于一个能够分解成子问题来解决, 且子问题的最优解能递推到最终问题的最优解的问题, 贪心算法是尤为有效的
		\item<3-> 一旦一个问题可以通过贪心法来解决, 那么贪心法一般是解决这个问题的最好办法
		\item<4-> 由于贪心法的高效性以及其所求得的答案比较接近最优结果, 贪心法也可以用作辅助算法或者直接解决一些要求结果不特别精确的问题
	\end{itemize}

\end{frame}


\begin{frame}[fragile]{特点}

	\begin{itemize}
		\item 最优子结构 (局部最优 \(\implies\) 整体最优)
		\item 高效
		\item 即使不是最优解, 一般来说也不会太差 (至少比随机方案好)
		\item 猜测简单(?), 证明困难
	\end{itemize}

\end{frame}


\begin{frame}[fragile]{证明方法}
	最基础最常用的方法有两种: 反证法与归纳法
\end{frame}
